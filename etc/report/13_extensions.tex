\chapter{Extensions}
The \verb|bnplib| library has several possible extensions:
\begin{itemize}
	\item New types of \verb|Hypers| classes can be implemented, for example ones containing hyper-priors for some of the parameters if the model.
	The algorithm must be modified accordingly, for instance by implementing the currently empty extra step such as \verb|update_hypers()|.
	Changes depend on the type of parameter for which a prior is used; for example, a prior on the total mass parameter involves different steps than a prior on the parameters of the base measure.
	For a general outline of the necessary changes, see \cite{neal} section 7.
	\item Hierarchies other than the NNIG and NNW can be created.
	This is enough to run \verb|Neal8| and \verb|Neal2| by passing the class name as parameter, provided that the \verb|Hierarchy| class has the appropriate interface.
	\item The same goes for other mixtures and algorithm.
	A good starting point for the latter would be the blocked Gibbs algorithm, which would make use of the \verb|update_weights()| substep.
	\item Conjugacy-dependent algorithms such as \verb|Neal2| can be further re-adapted to account for non-conjugacy, for example by using an Hamiltonian Monte Carlo sampler.
	\item Finally, a \emph{full generalization} of the library might be possible.
	That is, given the distributions of the likelihood, hyperparameters, etc, one might want an algorithm that works for the chosen specific model without needing and explicit implementation for it.
	This means, among other things, that one has to handle non-conjugacy for the general case.
	The main issue is that Stan distribution functions do not accept vectors of parameter values as arguments; thus, the updated values for distributions must be explicitly enumerated and given as arguments one by one to the Stan function.
	This requires to know in advance the number of parameters for all such distributions, which is impossible in the general case.
	Some advanced C++ techniques may be used to circumvent this hindrance, such as argument unpackers that transform a vector into a list of function arguments, and variadic templates, which are templates that accept any number of arguments.
	Theoretically, the latter would also allow the use of priors on the parameters of the hyper-prior itself, and so on, adding layers of uncertainty ad libitum.
	Although it is a hard task, we do think it is possible to achieve with reasonable effort.
\end{itemize}
