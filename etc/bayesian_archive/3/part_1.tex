\begin{frame}[c] %#01
	\begin{center}
		\huge \color{blue} Model
	\end{center}
\end{frame}


\begin{frame}{DP and DPM models} %#02
	Having observed the iid sample $\{y_i\}_i$, $i=1,\dots,n$:
	\begin{itemize}
		\item Dirichlet process model (discrete):
		\begin{align*}
		y_i | G &\iidsim G \\
		G &\sim DP(MG_0)
		\end{align*}
		\item Dirichlet process mixture (\textbf{DPM}) model (continuous):
		\begin{align*}
		y_i | G &\iidsim f_G(\cdot) = \int_\Theta f(\cdot|\boldsymbol\theta) \, G(\de\boldsymbol\theta) \\
		G &\sim DP(M G_0)
		\end{align*}
	\end{itemize}
\end{frame}


\begin{frame}{Equivalent formulations (1)} %#03
	\begin{itemize}
		\item (DPM) is equivalent to:
		\begin{align*}
		y_i | \boldsymbol\theta_i &\indsim f(\cdot|\boldsymbol\theta_i), \quad i=1,\dots,n \\
		\boldsymbol\theta_i | G &\iidsim G, \quad i=1,\dots,n \\ 
		G &\sim DP(M G_0)
		\end{align*}
		\item State $\forall i$: $\boldsymbol\theta_i$ latent variables (discrete)
	\end{itemize}
\end{frame}

\begin{frame}{Equivalent formulations (2)} %#04
	\begin{itemize}
		\item (DPM) is also equivalent to:
		\begin{align*}
		y_{i}|c_i,\boldsymbol\phi_1,\dots,\boldsymbol\phi_k &\indsim f(\cdot|\boldsymbol\phi_{c_{i}}), \quad i=1,\dots,n \\
		c_{i}|\mathit{\mathbf{p}}&\iidsim \sum_{j=1}^K\mathit{p_j} \delta_j(\cdot), \quad i=1,\dots,n \\
		\boldsymbol\phi_{c} & \iidsim G_{0}, \quad c=1,\dots,k \\
		\mathbf{p} &\sim \operatorname{Dir}(M/K,\dots,M/K) \\
		K &\to +\infty
		\end{align*}
		\item State $\forall i$: $c_i$ \textbf{allocations} to clusters
		\item State $\forall i$: $\boldsymbol\phi_{c_i}$ \textbf{unique values} for each cluster
		\item Only the finitely many $\boldsymbol\phi_{c}$ used are kept track of
	\end{itemize}
\end{frame}


\begin{frame}{Case study} %#05
	\begin{itemize}
		\item (DPM) with a Normal Normal-InverseGamma (\textbf{NNIG}) hierarchy:
		\begin{align*}
		y_i | \boldsymbol\theta_i &\indsim f(\cdot|\boldsymbol\theta_i), \quad i=1,\dots,n \\
		\boldsymbol\theta_i | G &\iidsim G, \quad i=1,\dots,n \\ 
		G &\sim DP(M G_0) \\
		f(y|\boldsymbol\theta)&=N(y| \mu ,\sigma^2)  \\
		G_0(\boldsymbol\theta|\mu_0,\lambda_0, \alpha_0, 	\beta_0)
		&=N\left(\mu | \mu_0 ,\frac{\sigma^2} {\lambda_0}\right) \times \text{Inv-Gamma}(\sigma^2|\alpha_0, \beta_0 )
		\end{align*}
		\item Latent variables: $\boldsymbol\theta = (\mu,\sigma)$
		\item State $\forall i$: $c_i$, $\boldsymbol\phi_{c_i}$
	\end{itemize}
\end{frame}


\begin{frame}[c] %#06
	\begin{center}
		\huge \color{blue} Algorithms
	\end{center}
\end{frame}


\begin{frame}[fragile]{General structure} %#07
\small
\ttfamily
\fontseries{l}\selectfont
template<template <class> class Hierarchy, \\
\ \ \ \ \ \ \ \ class Hypers, class Mixture> class Algorithm \\[10pt]
\fontseries{b}\selectfont
\ \ \ \ \ \ \ \ void step()\{ \\
\ \ \ \ \ \ \ \ \ \ \ \ sample\_allocations(); \\
\ \ \ \ \ \ \ \ \ \ \ \ sample\_unique\_values(); \\
\ \ \ \ \ \ \ \ \} \\[10pt]
\fontseries{l}\selectfont
\ \ \ \ \ \ \ \ void run()\{ \\
\ \ \ \ \ \ \ \ \ \ \ \ initialize(); \\
\ \ \ \ \ \ \ \ \ \ \ \ unsigned int iter = 0; \\
\ \ \ \ \ \ \ \ \ \ \ \ while(iter < maxiter)\{ \\
\fontseries{b}\selectfont
\ \ \ \ \ \ \ \ \ \ \ \ \ \ \ \ step(); \\
\fontseries{l}\selectfont
\ \ \ \ \ \ \ \ \ \ \ \ \ \ \ \ if(iter >= burnin)\{ \\
\ \ \ \ \ \ \ \ \ \ \ \ \ \ \ \ \ \ \ \ save\_iteration(iter); \\
\ \ \ \ \ \ \ \ \ \ \ \ \ \ \ \ \} \\
\ \ \ \ \ \ \ \ \ \ \ \ iter++; \\
\ \ \ \ \ \ \ \ \ \ \ \ \} \\
\ \ \ \ \ \ \ \ \} \\

\end{frame}


\begin{frame}{Auxiliary classes} %#08
	\begin{itemize}
		\item Specific common interface
		\item \texttt{Mixture} $\ \to \ $ \texttt{SimpleMixture}
		\item \texttt{Hypers} $\ \to \ $ \texttt{HypersFixedNNIG}
		\item \texttt{Hierarchy<Hypers>} $\ \to \ $ \texttt{HierarchyNNIG<Hypers>}
	\end{itemize}
\end{frame}


\begin{frame}[fragile]{\texttt{Neal8}} %#09
	\begin{itemize}
		\item Has a vector of $m$ \verb|aux_unique_values|
		\item \verb|initialize()|
		\item \verb|sample_allocations()|: for all observations $i=1,\dots,n$
		\begin{itemize}
			\item compute $\verb|card[c]| = n_{-i,c}$ for all clusters $c=1,\dots,k$
			\item if $c_i$ is a singleton, move $\boldsymbol\phi_{c_i}$ to \verb|aux_unique_values[0]|
			\item draw all (other) \verb|aux_unique_values| iid from $G_0$
			\item draw a new value $c$ for $c_i$ according to:
			\begin{align*}
				\small
				\hspace{-30pt}
				\PP(c_{i}=c | \boldsymbol c_{-i}, y_{i}, \boldsymbol\phi_{1},\dots,\boldsymbol\phi_{h}) \propto
				\begin{cases}
				\dfrac{n_{-i,c}}{n-1+M}f(y_{i}|\boldsymbol\phi_{c}), & \mbox{for } 1 \leq c \leq k^{-} \\[5pt]
				\dfrac{M/m}{n-1+M}f(y_{i}|\boldsymbol\phi_{c}), & \mbox{for } k^{-}+1 < c \leq h
				\end{cases}
			\end{align*}
			with $k^-$ unique values excluding $c_i$ and $h = k^-+m$
			\item update \verb|card| and \verb|allocations| (4 cases)
		\end{itemize}
		\item \verb|sample_unique_values()|: for all clusters $c = 1,\dots,k$
		\begin{itemize}
			\item build \verb|curr_data| that contains all observations in cluster $c$
			\item draw $\boldsymbol\phi_c$ from its posterior distribution given \verb|curr_data|
		\end{itemize}
	\end{itemize}
\end{frame}


\begin{frame}[fragile]{\texttt{Neal2}} %#10
	\begin{itemize}
		\item For conjugate models only, e.g. (DPM)+(NNIG)
		\item \verb|initialize()|
		\item \verb|sample_allocations()|: for all observations $i=1,\dots,n$
		\begin{itemize}
			\item compute $\verb|card[c]| = n_{-i,c}$ for all clusters $c=1,\dots,k$
			\item draw a new value $c$ for $c_i$ according to:
			\begin{align*}
				\hspace{-30pt}
				\text{If $c=c_j$ for some $j$: }
				\PP(c_{i}=c | \boldsymbol c_{-i}, y_{i},\boldsymbol\phi) &\propto \frac{n_{-i,c} }{n-1+M} f(y_{i}|\boldsymbol\phi_{c}) \\
				\PP(c_{i}\neq c_{j} \text{ for all } j | \boldsymbol c_{-i}, y_{i},\boldsymbol\phi) &\propto \frac{M }{n-1+M} \int_{\Theta} f(y_{i}|\boldsymbol\theta) \, G_0(\de\boldsymbol\theta)
			\end{align*}
			\item if the latter, draw a new $\boldsymbol\phi_c$ from its posterior given $y_i$
			\item update \verb|card| and \verb|allocations| (4 cases)
		\end{itemize}
		\item \verb|sample_unique_values()|: for all clusters $c = 1,\dots,k$
		\begin{itemize}
			\item build \verb|curr_data| that contains all observations in cluster $c$
			\item draw $\boldsymbol\phi_c$ from its posterior distribution given \verb|curr_data|
		\end{itemize}
	\end{itemize}
\end{frame}
