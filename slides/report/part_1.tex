\section{Introduction}
This report presents the development of a C++ Bayesian nonparametric library containing Marcov Chain sampling methods for density estimation and clustering. 

TODO: add

\section{The model}
In a Bayesian nonparametric setting, we focused on the Dirichlet process (DP) and its extensions, one of the most widely used priors due to its flexibility and computational ease.

\subsection{Dirichlet Process}
Let $M>0$ and $G_0$ be a probability measure defined on $S$. A DP with parameters $(M,G_0)$ is a random probability measure $G$ defined on $S$ which assigns probability $G(B)$ to every set $B$ such that for each finite partition ${B_1,\dots,B_k}$ of $S$, the joint distribution of the  vector $(G(B_1),\dots,G(B_k))$ is the Dirichlet distribution with parameters
\begin{align*}
(MG_0(B_1),\dots,MG_0(B_k)).
\end{align*}
The parameter $M$ is called the precision or total mass parameter, $G_0$ is
the centering measure, and the product $MG_0$ the base measure
of the DP. \\
Having observed the independent and identically distributed sample $\{y_1,\dots,y_n\}$, the basic DP model takes the following form:
\begin{align}
		y_i | G &\iidsim G, \quad i=1,\dots,n \\
		G &\sim DP(MG_0)
\end{align}
A key property is that the DP is conjugate with respect to independent and identically distributed sampling, so that the posterior base distribution is a weighted average of the prior base distribution $G_0$ and the empirical distribution of the data, with the weights controlled by $M$:
\begin{align}
	G | y_1,\dots,y_n \sim DP(M G_0 + \sum_{i=1}^n \delta_{y_i}).
\end{align}
Moreover, the marginal distribution will be the product of the sequence of increasing conditionals:
\begin{align}
	p(y_1,\dots,y_n)= p(y_1)\prod\limits_{i=2}^{n} p(y_i|y_1,\dots,y_{i-1}),
\end{align}
with $y_1 \sim G_0$ and for $i=2,3,\dots$ defined as:
\begin{align}
	p(y_i|y_1,\dots,y_{i-1})= \frac{1}{M+i-1}\sum_{h=1}^{n-1} \delta_{y_h}(y_i) +\frac{M}{M+i-1} G_0(y_i).
\end{align}
An important property of the DP is the discrete nature of $G$. Since it is a discrete
random probability measure, we can always write $G$ as a weighted sum of point masses.
A useful consequence of this property is its stick-breaking representation, i.e. $G$ can be written as:
\begin{align}
	G(\cdot) = \sum_{k=1}^{+\infty} w_k \delta_{m_k} (\cdot),
\end{align}
with $m_k \iidsim G_0$ for $k\in\mathbb{N}$ and the random weights constructed as $w_k =v_k\prod\limits_{l<k} (1-v_l)$ where $v_k \iidsim Be(1,M)$. \\
In many applications in which we are interested in a continuous density estimation this discreteness can represent a limitation.
Oftentimes a Dirichlet Process Mixture (DPM) model is used, where the DP random measure is the mixing measure for the parameters of a parametric continuous kernel function.

\subsection{Dirichlet Process Mixture Model}
Let $\Theta$ be a finite-dimensional parameter space and $G_0$ a probability measure on $\Theta$.
The DPM model convolves the densities $f_\theta$ from a parametric family $F =\{f_\theta| \theta \in \Theta \}$ using the DP as mixture weights.
The obtained model has the following form:
\begin{align}\label{dpm-1}
			y_i | G &\sim f_G(\cdot) = \int_\Theta f_{\theta}(\cdot) \, G(\de\theta), \quad i=1,\dots,n \\
			G &\sim DP(M G_0) \nonumber
\end{align}
An equivalent hierarchical model is:
\begin{align}\label{dpm-2}
	y_i | \theta_i &\indsim f_{\theta_i}(\cdot), \quad i=1,\dots,n \\
	\theta_i | G &\iidsim G, \quad i=1,\dots,n \nonumber \\ 
	G &\sim DP(M G_0) \nonumber
\end{align}
where the \textit{latent variables} $\theta_i$ are introduced, one per unit.
Since $G$ is discrete, we know that two independent draws $\theta_i$ and $\theta_j$ from $G$ can be equal with positive probability.
In this way the DPM model induces a probability model on clusters and an object of interest that derives from this model is the partitioning induced by the clustering, as well as the density estimation. \\
Considering $n$ data units, each $\theta_i$ will have one of the $k$ unique values $\phi_{j}$. An estimation of the number of the unique values is $M\log(n) \ll n$. Defining  $c_i$ the \textit{allocation} parameters to the clusters such that $c_i = j$ if $\theta_i = \phi_j$, the model can be thought as the limit as $K$ goes to infinity of
finite mixture model with $K$ components:
\begin{align}\label{dpm-disc}
	y_{i}|\mathbf{\phi},c_{i} &\sim f_{\phi_{c_{i}}}(\cdot) \\
   	c_{i}|\mathit{\mathbf{p}}&\sim \sum_{k=1}^K\mathit{p_k} \delta_k(\cdot) \nonumber \\
    \phi_{c} & \sim G_{0} \nonumber \\
    \mathbf{p} &\sim \operatorname{Dir}(M/K,\dots,M/K) \nonumber
\end{align}
where $\mathbf{p}=(p_1,\dots,p_K)$ represents the mixing proportion for the classes and each $\theta$ is characterized by the latent class $c$ and the corresponding parameters $\phi_c$.

\subsubsection{Normal Normal-InverseGamma model} \label{nnig}
A very common choice for the DPM model is the gaussian mixture model, also known as Normal - Normal-InverseGamma (Normal-NIG) model, opting for a Normal kernel and the conjugate Normal-InverseGamma as base measure $G_0$. With $\theta=(\mu,\sigma^2)$, we have:

\begin{align}
            	f_\theta(y)&=N(y| \mu ,\sigma^2)  \\
            	G_0(\theta|\mu_0,\lambda_0, \alpha_0, \beta_0)&=N\left(\mu | \mu_0 ,\frac{\sigma^2} {\lambda_0}\right)\text{Inv-Gamma}(\sigma^2|\alpha_0, \beta_0 )
\end{align}
Thanks to conjugacy, the predictive distribution for a new observation $\widetilde{y}$ can be computed analitically, finding a Student's $t$ (see \cite{integral} section 2.3):
\begin{align}
      p(\widetilde{y}|\mu_0,\lambda_0, \alpha_0, \beta_0)= \int_\Theta f_\theta(\widetilde{y})G_0(\de\theta)= \frac{1}{\widetilde{\sigma}} \text{t}\left(\frac{\widetilde{y}-\widetilde{\mu}}{\widetilde{\sigma}} ,|\widetilde{v}\right)        	
\end{align}
where the following parameters are set:
\begin{align*}
	\widetilde{v}=2 \alpha_0, \widetilde{\mu}=\mu_0, \\
	\widetilde{\sigma}^2= \frac{\beta_0(\lambda_0+1)}{\alpha_0 \lambda_0}
\end{align*}
The posterior distribution is again a Normal-InvGamma:
\begin{align}
          p(\theta|y,\mu_0,\lambda_0, \alpha_0, \beta_0)=N\left(\mu | \mu_n ,\frac{\sigma^2} {\lambda_0 + n}\right)\text{Inv-Gamma}(\sigma^2|\alpha_n, \beta_n )
\end{align}
with $\mu_n=\frac{\lambda_0 \mu_0 \bar{y} + n}{\lambda_0 + n}$ , $\alpha_n= \alpha_0 + \frac{n}{2} $ and $\beta_n= \beta_0 + \frac{1}{2}\sum_{i=1}^{n} (y_i-\bar{y})^2 + \frac{\lambda_0 n(\bar{y}-\mu_0)^2}{2(\lambda_0 + n)}$.

\section{Algorithms}

For the task of density estimation, we investigated several Markov chain methods to sample from the posterior distribution of a Dirichlet process mixture model. \\
Starting from the hierarchical model $(7)$, a first direct approach is simply drawing values for each $\theta_i$ from its conditional given the data and the other $\theta_j$. %TODO ref
However, as we discussed before, there is high probability for ties among them which can lead to slow convergence, since the $\theta$ values are not updated for more than one observation simultaneously. \\
For this reason, special attention was paid to the three of them we presents in this chapter.
They are Gibbs samplers with a similar base structure, sharing the two steps for the sampling of the allocations $\mathbf{c}$ and of the unique values $\mathbf{\phi_c}$. \\
Moreover, all methods can be extended with additional steps for hierarchical extensions.
We can think to place priors to hyperparamaters of the centering measure $G_0$ or to the total mass $M$.

\subsection{Neal's Algorithm 2 (\texttt{Neal2})}
In order to speed up the convergence in case of ties, Neal first proposed (see \cite{neal} section 3) a more efficient Gibbs sampling method based on the discrete model $(8)$, where the mixing proportions $\textbf{p}$ have been integrated out. %TODO ref
Assuming that the current state of Markov chain is composed of $(c_1,\dots,c_n)$  and the component parameters $\phi_c$ for all $c$, the Gibbs sampler first draws values for each $c_i$ given the following conditional probabilities:
\begin{align}
			\PP(c_{i}=c | \mathbf c_{-i}, y_{i},\boldsymbol\phi) \propto \frac{n_{-i,c} + M/{K}}{n-1+M} f_{\phi_{c}}(y_{i}) 
\end{align}
where $n_{-i,c}$ is the number of $c_j$ equal to $c$ excluding $c_i$.
Consequently, the sampler draws a new value for each $\phi_c$ given the data belonging to that class.
The passage to the infinite case is done taking the limit as $K$ goes to infinity in the conditional distribution of $c_i$ which becomes as follows:
\begin{align}
	\PP(c_{i}=c | \mathbf c_{-i}, y_{i},\boldsymbol\phi) &\propto \frac{n_{-i,c} }{n-1+M} f_{\phi_{c}}(y_{i}) \\
	\PP(c_{i}\neq c_{j} \text{ for all } j | \mathbf c_{-i}, y_{i},\boldsymbol\phi) &\propto \frac{M }{n-1+M} \int_{\Theta} f_{\phi}(y_{i}) \, G_0(\de\phi) \nonumber 
\end{align}
and considering only the $\phi_c$ associated with some observation, keeping the sampling finite and thus computationally feasible.
At this point the algorithm works iteratively by sampling $c$ and $\phi$.
For each observation $i$, $c_i$ is updated according to its conditional distribution.
It can be set either to one of the other components currently associated with some observation, or to a new mixture component.
If the new value of $c_i$ is different from all the other $c_j$, a value for $\phi_{c_i}$ is drawn from the posterior distribution $H_i$, based on the prior $G_0$ and the single observation $y_i$.
Then for all the classes the  sample for $\phi_c$ is done considering the posterior distribution based on the prior and all the observations belonging to the specific class.
The probability of setting $c_i$ to a new component involves the integral $\int_{\Theta} f_{\phi}(y_{i}) \, G_0(\de\phi)$, which is difficult in the non-conjugate case, as well as the sample from the posterior $H_i$.

\subsection{Neal's Algorithm 8 (\texttt{Neal8})}
To handle non-conjugate priors, Neal proposed (see \cite{neal} section 6) a second Markov chain sampling procedure where the state is extended by the addition of auxiliary parameters.
This technique allows to update the $c_i$ while avoiding the integration with respect to $G_0$. \\
In this case the prior for $c_i$ is:
\begin{align}
            \hspace{-25pt}
            \vspace{-12pt}
                \text{If $c=c_j$ for some $j$: } \PP(c_i=c | \mathbf c_{-i}) &= \frac{n_{-i,c}}{n-1+M}   \\
                \PP(c_{i}\neq c_{j} \text{ for all } j) &=\frac{M }{n-1+M}                  \nonumber
\end{align}	
where the probability of selecting a new component is evenly split among the $m$ auxiliary components.
Whilst maintaining the same structure as the \verb|Neal2|, the \verb|Neal8| is composed of two steps, where the components of the Markov chain state $(c,\phi)$ are repeatedly sampled.
The first step scans all the observations and evaluates each $c_i$.
If it is equal to another $c_j$, then all the auxiliary variables are drawn from $G_0$.
If the corresponding cluster is a singleton, then it is linked to one of the auxiliary variable with the corresponding value of $\phi_c$, while the others are drawn as before from $G_0$.
Then, $c_i$ is updated according to the following conditional probabilities:
\begin{align}
	\PP(c_{i}=c | \mathbf c_{-i}, y_{i}, \phi_{1},\dots,\phi_{h}) \propto \begin{cases}  \frac{n_{-i,c}}{n-1+M}f_{\phi_{c}}(y_{i}), & \mbox{for } 1 \leq c \leq k^{-} \\ \frac{M/m}{n-1+M}f_{\phi_{c}}(y_{i}), & \mbox{for } k^{-}+1 < c \leq h,
	\end{cases}
\end{align}
indicating with $k^{-}$ the number of distinct $c_j$ excluding the current $c_i$ and setting $h=k^{-}+m$. \\
Once all the $\phi_c$ that are no longer associated with any observation are discarded, the algorithm proceeds with the sampling for $\phi_c$ for all the classes.

\subsection{Blocked Gibbs}
Another Gibbs Sampling method applicable in the considered Bayesian hierarchical models is the one proposed by Ishwaran and James (see \cite{james} section 5), where the prior $P$ is assumed to be a finite dimensional stick-breaking measure allowing in this way to update blocks of parameters.
A key point of the method is that it does not marginalize over the prior, instead, grouping more variables together, it samples from their joint distribution conditioned on all other variables. \\
It needs to draw from the conditionals:
\begin{align}
	\boldsymbol\phi &\sim \Lc(\boldsymbol\phi | \mathbf{c}, \mathbf{y}) \nonumber \\
	\mathbf{c} &\sim \Lc (\mathbf{c}| \boldsymbol\phi,\mathbf{p}, \mathbf{y}) \nonumber \\
	\mathbf{p} &\sim \Lc (\mathbf{p}| \mathbf{c}) \nonumber
\end{align}
The draw for the unique values can be easily handled also in the non-conjugate case by applying standard Markov chain Monte Carlo methods.


TODO: REFERENCEs